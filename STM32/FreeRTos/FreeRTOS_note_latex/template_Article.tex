\documentclass[11pt,a4paper]{ctexart}

\usepackage{subfigure}
\usepackage[graphicx]{realboxes}
\usepackage{listings}
\usepackage{xcolor}
\usepackage{amsmath}

% (1) choose a font that is available as T1
\usepackage{lmodern}
% (2) specify encoding
\usepackage[T1]{fontenc}
% (3) load symbol definitions
\usepackage{textcomp}

\usepackage{hyperref}

\usepackage{graphicx}

\usepackage{tabularx}
\usepackage{array}


\hypersetup{hidelinks,
	colorlinks=true,
	allcolors=black,
	pdfstartview=Fit,
	breaklinks=true}


\definecolor{mygrey}{rgb}{0.945,0.945,0.945}
\definecolor{myred}{rgb}{1, 0.49, 0.63}
\definecolor{myblue}{rgb}{0,0.36, 0.67}

\lstset{
	breaklines=true,
	basicstyle=\fontspec{Consolas},
	columns=fixed,       
	numbers=left,                                        % 在左侧显示行号
	numberstyle=\tiny\color{gray},                       % 设定行号格式
	frame=none,                                          % 不显示背景边框
	backgroundcolor=\color[RGB]{245,245,244},            % 设定背景颜色
	keywordstyle=\color[RGB]{40,40,255},                 % 设定关键字颜色
	numberstyle=\footnotesize\color{darkgray},           
	commentstyle=\it\color[RGB]{0,96,96},                % 设置代码注释的格式
	stringstyle=\rmfamily\slshape\color[RGB]{128,0,0},   % 设置字符串格式
	showstringspaces=false,                              % 不显示字符串中的空格
	language=c++,                                        % 设置语言
}
%\def\inline{\lstinline[basicstyle=\fontspec{微软雅黑},keywordstyle={}]}


%opening
\title{FreeRTOS}
\author{Liam}
\date{\today}

\begin{document}

\maketitle

\begin{abstract}
Something to add\dots
\end{abstract}

\section{裸机系统与多任务系统}
\subsection{裸机系统}
\subsubsection{轮询系统}
\textcolor{myblue}{定义}:
\par
轮询系统即是在裸机编程的时候,初始化好相关的硬件,然后将主程序在一个\textbf{死循环}里面不断地循环,顺序处理各种事件。(当有要检测外部信号的时候实时性较差)
\subsubsection{前后台系统}
\textcolor{myblue}{定义}:
\par
前后台系统即是在轮询系统上加上了\textbf{中断},外部事件的响应在终端里面完成,事件处理还是在轮询系统中完成。
中断被称为\textbf{前台},\colorbox{mygrey}{\color{myred}\lstinline|main()|}函数被称为\textbf{后台}。
~\\
\par
相对于轮询系统,前后台系统确保事不会丢失,加上中断可嵌套,大大提高程序的实时响应功能。
\subsection{多任务系统}
相对于前后台系统,多任务系统的事件响应都是在中断中完成,但是事件的处理实在任务中完成的。

\begin{quote}
	多任务系统中,\textbf{任务}与\textbf{中断}一样,也具有优先级,优先级高的任务会被优先执行。当一个紧急事件在中断中被标记之后,如果事件对应的任务的
	优先级足够高,就会立即得到相应。
\end{quote}
\par
在多任务系统中,根据程序的功能,我们把程序中体分割成\textbf{分割成一个个独立的,无限循环不能返回的小程序}----任务。各个任务独立互不干扰,
各自具备自身的优先级,它由操作系统调度管理。

\begin{center}
	\begin{tabular}{|c|c|c|c|c|}
		\hline
		& 模型 & 事件响应 & 事件处理 & 特点 \\ \hline
		& 轮询系统 & 主程序 & 主程序 & 轮询响应并处理事件 \\ \hline
		& 前后台系统 & 中断 & 主程序 & 实时响应事件,轮询处理 \\ \hline
		& 多任务系统 & 中断 & 任务 & 实时响应并处理事件 \\ \hline
	\end{tabular}
\end{center}

\subsection{数据结构--列表与列表项}
\textcolor{myblue}{列表和列表项}对应C语言中的链表和节点。
~\\
\textcolor{myred}{链表复习}
\subsubsection{单向列表}
节点本身包含一个指针,用于只想后一个节点,并且可以携带一些私有信息。

列表一般包含一个\textbf{节点计数器},\textbf{节点插入和删除操作器}
\subsubsection{双向列表}
节点中将会有两个节点指针,一个指向头一个指向尾部。

\subsection{FreeRTOS链表实现}
FreeRTOS中链表的实现,均在\colorbox{mygrey}{\color{myred}\lstinline|list.h|}和\colorbox{mygrey}{\color{myred}\lstinline|list.c|}中。示意图:
\begin{center}
	\begin{tabular}{|c|}
		\hline
		xLISt\textunderscore tIEM \\ \hline
		pxNext \\ \hline
		pxPrevious \\ \hline
		pvOwner \\ \hline
		pvContainer \\ \hline
	\end{tabular}
\end{center}

\begin{lstlisting}
struct xLIST_ITEM
{
	TickType_t xItemValue;           //辅助值,帮助节点进行顺序排列
	struct xLIST_ITEM * pxNext;      //指向下一个节点
	struct xLIST_ITEM * pxPrevious;  //指向上一个节点
	void * pvOwner;                  //指向拥有该节点的内核对象
	void * pvContainer;              //指向该节点所在链表
};
typedef struct xLIST_ITEM ListItem_t;
\end{lstlisting}
\begin{quote}
	在FreeRTOS中,凡是涉及数据类型的地方,标准的C数据类型用{\color{myred}\lstinline|typedef|}重新设置在{\color{myred}\lstinline|portmacro.h|}
\end{quote}

\subsubsection{实现根节点}
\begin{lstlisting}
	typedef struct xLIST
	{
		UBaseType_t uxNumberOfItems;        //链表节点计数器
		ListItem_t * pxIndex;               //链表节点索引指针
		MiniListItem_t xListEnd;            //链表最后一个节点
	} List_t;
\end{lstlisting}

链表根节点初始化
\begin{lstlisting}
void vListInitialise(List_t * const pxList)
{
	pxList -> pxIndex = ( ListItem_t *) &(pxList->xListEnd);        //将链表索引指针指向最后一个节点
	pxList->xListEnd.xItemValue = portMAX_DELAY;                    //辅助排序设置为最大
	pxList->xListEnd.pxNext = (ListItem * ) &(pxList ->xListEnd);   //最后一个节点的pxNext和previous指向自己
	pxList->xListEnd.pxPrevious = (ListItem_t * ) &(pxList->xListEnd);
	pxList->uxNumberOfItems = (UBaseType_t) 0U;   //节点计数值为0,表示列表为空。
}
\end{lstlisting}

\section{任务的定义与任务切换}
\subsection{任务}
在多任务系统中,我们根据功能不同,将整个系统分割为一个个独立且无法返回的函数。
\subsubsection{定义任务栈}
在多任务系统中,要为每一个任务独立分配\textbf{栈空间}
\begin{lstlisting}
	#define TASK1_STACK_SIZE
	StackType_t Task1Stack[TASK1_STACK_SIZE];
\end{lstlisting}
\subsubsection{定义任务控制块}
在裸机系统中,任务是CPU按照顺序执行的;在多任务系统中是由系统调度的。系统为顺利调取任务,为每个任务都定义了一个\textbf{任务控制块},
\begin{lstlisting}
typedef struct tskTaskControlBlock
{
	volatile StackType_t *pxTopOfStack;
	ListItem_t xStateListItem;  //这是一个内置在TCB控制块中的链表节点。
	StackType_t *pxStack;
	char pcTaskName[configMAX_TASK_NAME_LEN];
}tskTCB;
typedef tskTCK TCB_t;
\end{lstlisting}
\subsubsection{实现任务创建函数}
任务栈、函数实体和任务控制块都需要最终联系起来才能由系统统一调度。

\end{document}
