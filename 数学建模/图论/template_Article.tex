\documentclass[12pt,a4paper]{ctexart}

\usepackage{subfigure}
\usepackage[graphicx]{realboxes}
\usepackage{listings}
\usepackage{xcolor}
\usepackage{amsmath}

% (1) choose a font that is available as T1
\usepackage{lmodern}
% (2) specify encoding
\usepackage[T1]{fontenc}
% (3) load symbol definitions
\usepackage{textcomp}

\usepackage{hyperref}

\usepackage{graphicx}

\hypersetup{hidelinks,
	colorlinks=true,
	allcolors=black,
	pdfstartview=Fit,
	breaklinks=true}


\definecolor{mygrey}{rgb}{0.945,0.945,0.945}
\definecolor{myred}{rgb}{1, 0.49, 0.63}

\lstset{
breaklines=true,
 basicstyle=\fontspec{Consolas},
 columns=fixed,       
 numbers=left,                                        % 在左侧显示行号
 numberstyle=\tiny\color{gray},                       % 设定行号格式
 frame=none,                                          % 不显示背景边框
 backgroundcolor=\color[RGB]{245,245,244},            % 设定背景颜色
 keywordstyle=\color[RGB]{40,40,255},                 % 设定关键字颜色
 numberstyle=\footnotesize\color{darkgray},           
 commentstyle=\it\color[RGB]{0,96,96},                % 设置代码注释的格式
 stringstyle=\rmfamily\slshape\color[RGB]{128,0,0},   % 设置字符串格式
 showstringspaces=false,                              % 不显示字符串中的空格
 %language=python,                                        % 设置语言
}
%\def\inline{\lstinline[basicstyle=\fontspec{微软雅黑},keywordstyle={}]}
%opening
\title{The Note of Graph Theory}
\author{Liam}
\date{\today}

\begin{document}

\maketitle

\begin{abstract}
I hate definition!
\end{abstract}

\section{definition}
\begin{center}
    \colorbox{mygrey}{\color{myred}\textbf{simple graph}:}
    \begin{itemize}
        \item a non-empty finite set $V(G)$ of elements \textbf{vertices}
        \item a finnite set $E(G)$ of distinct unordered pairs of deistinct elements \textbf{edges}
        \item at most \textbf{one edge} joining a given pair of vertices
    \end{itemize}
\end{center}
\begin{center}
    \colorbox{mygrey}{\color{myred}\textbf{general graph}:}
    \begin{itemize}
        \item multiple edges
        \item loops
    \end{itemize}
\end{center}
\begin{center}
    \colorbox{mygrey}{\color{myred}graph}
    \begin{itemize}
        \item a non-empty finite set $V(G)$ of elements \textbf{vertices}
        \item a finnite family $E(G)$ of distinct unordered pairs of not necessarily deistinct elements \textbf{edges}
        \item the use of \textcolor{myred}{family} permits the existence of multiple edge
        \item vertex-set  edge-family
    \end{itemize}
\end{center}
$\{v,w\}$ is said to join the vertices v , w and is again abbreviated to $vw$. Example : $vv$ 

\begin{center}
    \colorbox{mygrey}{\color{myred}isomorphism}
    \begin{itemize}
        \item one-one correspondence between the vertices of $G_1$ and those of $G_2$ such that the number of edges joining any two vertices of $G_1$ equals the number of edges joining the corresponding vertices of $G_2$
        \item the \textbf{unlabelled graphs} and \textbf{labelled graphs} is different
    \end{itemize}
\end{center}

\begin{center}
    \colorbox{mygrey}{\color{myred}connected graphs}
    \begin{itemize}
        \item A graph is connected of it cannot be expressed as a union of graghs and disconnected otherwise.
    \end{itemize}
\end{center}

\begin{center}
    \colorbox{mygrey}{\color{myred}Adjacency and degrees}
    \begin{itemize}
        \item the $v,w$ is \textbf{adjacency} if there is an edge vw joining them. and $v,w$ are incident with such edge
        \item degree of a vetex is the number of edges incident with itself.
        \item A vetex of degree 0 is an isolated vertex 
        \item A vetex of degree 1 is an end vertex 
    \end{itemize}
\end{center}
The \textbf{degree sequence} of a graph consists of degrees written in \textbf{increasing order}

\begin{center}
    \colorbox{mygrey}{\color{myred}Handshaking lemma}\colorbox{mygrey}{In any graph the sum of all the vertex-degrees}
    \colorbox{mygrey}{is an even number.~~~~~~~~~~~~~~~~~~~~~~~~~~~~~~~~~~~~~~~~~~~~~~~~~~~~~~~~~~~~~~~}
\end{center}
\colorbox{mygrey}{\color{myred}Corollary}\colorbox{mygrey}{In any graph the number of vertices of odd degree is even.}
\begin{center}
    \colorbox{mygrey}{\color{myred}Subgraphs}
    \begin{itemize}
        \item A graph is a \textbf{subgraph} of graph G if each of its vertices belongs to $V(G)$ and each of its edges belongs to $E(G)$
        \item we can obtain subgraphs by deleting edges and vertices.
        \item $G-F$
    \end{itemize}
\end{center}
\end{document}
