\documentclass[11pt,a4paper]{ctexart}

\usepackage{subfigure}
\usepackage[graphicx]{realboxes}
\usepackage{listings}
\usepackage{xcolor}
\usepackage{amsmath}

% (1) choose a font that is available as T1
\usepackage{lmodern}
% (2) specify encoding
\usepackage[T1]{fontenc}
% (3) load symbol definitions
\usepackage{textcomp}

\usepackage{hyperref}

\usepackage{graphicx}


\hypersetup{hidelinks,
	colorlinks=true,
	allcolors=black,
	pdfstartview=Fit,
	breaklinks=true}


\definecolor{mygrey}{rgb}{0.945,0.945,0.945}
\definecolor{myred}{rgb}{1, 0.49, 0.63}

\lstset{
breaklines=true,
 basicstyle=\fontspec{Consolas},
 columns=fixed,       
 numbers=left,                                        % 在左侧显示行号
 numberstyle=\tiny\color{gray},                       % 设定行号格式
 frame=none,                                          % 不显示背景边框
 backgroundcolor=\color[RGB]{245,245,244},            % 设定背景颜色
 keywordstyle=\color[RGB]{40,40,255},                 % 设定关键字颜色
 numberstyle=\footnotesize\color{darkgray},           
 commentstyle=\it\color[RGB]{0,96,96},                % 设置代码注释的格式
 stringstyle=\rmfamily\slshape\color[RGB]{128,0,0},   % 设置字符串格式
 showstringspaces=false,                              % 不显示字符串中的空格
 %language=python,                                        % 设置语言
}
%\def\inline{\lstinline[basicstyle=\fontspec{微软雅黑},keywordstyle={}]}

%opening
\title{非线性规划}
\author{Liam}
\date{\today}

\begin{document}
\maketitle



\section{非线性规划模型}
如果目标函数或者约束条件包含非线性函数,则称为非线性规划问题。

\subsection{无约束极值问题}
\subsubsection{无约束极值问题的数值解}
使用命令:
\par 
\colorbox{mygrey}{\color{myred}\lstinline|[x,fval]=fminunc(fun,x0,options)|}
\par
\colorbox{mygrey}{\color{myred}\lstinline|[x,fval]=fminsearch(fun,x0,options)|}



\subsection{约束极值问题}
带约束条件的极值问题成为约束极值问题。一般都是将非线性化简为\textbf{线性问题} ,约束问题化简为\textbf{无约束问题}。
\subsubsection{二次规划}
如果某个非线性规划的目标函数为子标量的二次函数,约束为线性。
\[
\begin{aligned}
    & \min \qquad \frac{1}{2}x^THx+f^Tx , \\
    & \text{s.t.} \qquad \left\{
        \begin{array}{rl}
            & \boldsymbol{A}\boldsymbol{x} \le b,\\
            & \boldsymbol{A}eq*\boldsymbol{x}=beq,\\
            &lb \le \boldsymbol{x} \le ub.
        \end{array} \right.
\end{aligned}  
\]
\lstset{
  literate={\_}{}{0\discretionary{\_}{}{\_}}%
}
求解二次规划的命令为
\par
\colorbox{mygrey}{\color{myred}\lstinline|[x,fval] = quadprog(H,f,A,b,Aeq,beq,lb,ub,x0,options)|}
\subsubsection{罚函数法}
罚函数法又称乘子法,是指将有约束最优化问题转化为求解无约束最优化问题:其中$\sigma$ 为足够大的正数, 起"惩罚"作用, 称之为罚因子, $F(x, \sigma )$ 称为罚函数。内部罚函数法也称为障碍罚函数法。
\par
外罚函数法:适用于含等式或者不等式约束的情形。思想是,在目标函数上加上惩罚项,该惩罚项由约束函数构造,
在可行域中\textbf{不予惩罚},但是在不可行域中\textbf{施加惩罚},从而使目标函数在不可行域中的值变得很大,
此时可以去掉约束项,看成一个\textbf{无约束优化问题}。如果最优解出现在可行域的边界上,随着惩罚力度的加大,无约束问题的最优解会在可行域外部逼近原问题最优解。
外罚函数法的主要缺点:
\begin{enumerate}
  \item \textbf{近似的最优解往往不是可行解}。从上面的例子很容易看出,最优解如果在边界上,那么近似的最优解是从可行域外部逼近的,这在一些实际应用场合是不能接受的(函数边界外面没有定义).
  \item \textbf{惩罚力度往往需要趋于无穷才能让近似的最优解逼近真正的最优解,但随着的增大,海森阵容易趋于病态}。尤其在用数值迭代的过程中造成不稳定的现象(所谓的梯度悬崖),甚至是无法求解。
\end{enumerate}
~\\
\par
内罚函数法:
仅适用于含有不等式的情形,也在目标函数上加上惩罚项(障碍项),障碍项作用在可行域内部,特别是在靠近边界时候目标函数$\rightarrow \infty$,这相当于在可行域边界上筑起一道高墙,让迭代点跳不出去,此时可以去掉约束项,看成一个无约束
优化问题。若最优值出现在可行域的边界上,由于障碍项会影响可行域内部的值,所以要减少障碍力度,让其尽量不影响原目标函数在可行域
内部形成的值,但同时还要在边界上保持高墙,当障碍力度充分小的时候,无约束最优化问题会在可行域内部逼近原问题最优解。
\par
内罚函数的优点:
\begin{enumerate}
  \item 每个近似最优解都是可行解(因为迭代点始终处于可行域内部)
\end{enumerate}
缺点:
\begin{enumerate}
  \item 仅适用于不等式约束
  \item 寻找初始可行点可能不容易
  \item 障碍因子r不断减小也会导致海森阵趋于病态(梯度悬崖)
  \item 在数值求解过程中造成很大麻烦
\end{enumerate}
如果非线性规划问题要求实时算法,则可以使用罚函数方法,但计算精度低。
\subsubsection{Matlab求解约束极值问题}
\[ 
  \begin{aligned}
    \left\{
      \begin{array}{rl}
        & \text{fminbnd(fun,x1,x2,options)    单变量非线性区间}[x_1,x_2] \\
      & \text{fmincon} \\
      & \text{quadprog} \\
      & \text{fseminf} \\
      & \text{fminimax} 
      \end{array}
    \right.
  \end{aligned}
\]
\colorbox{mygrey}{\color{myred}\lstinline|[x,fval]=fseminf(fun,x0,ntheta,seminfcon,A,b,Aeq,beq,lb,ub)|}

\subsubsection{飞行管理问题}



\end{document}
